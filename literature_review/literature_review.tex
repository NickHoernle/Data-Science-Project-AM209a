\documentclass[11pt]{article}
\usepackage{fullpage,amsmath,amsfonts,mathpazo,microtype,nicefrac,graphicx,verbatimbox,listings,hyperref,enumitem,amssymb,float,fancyhdr}

% Margins
% \topmargin=-0.45in
% \evensidemargin=0in
% \oddsidemargin=0in
% \textwidth=6.5in
% \textheight=9.0in
% \headsep=0.25in 

% \linespread{1.1} % Line spacing

% % Set up the header and footer
% \pagestyle{fancy}
% \lhead{\hmwkAuthorName} % Top left header
% \chead{\hmwkClass\ (\hmwkClassInstructor\ \hmwkClassTime): \hmwkTitle} % Top center header
% \rhead{\firstxmark} % Top right header
% \lfoot{\lastxmark} % Bottom left footer
% \cfoot{} % Bottom center footer
% \rfoot{Page\ \thepage\ of\ \pageref{LastPage}} % Bottom right footer
% \renewcommand\headrulewidth{0.4pt} % Size of the header rule
% \renewcommand\footrulewidth{0.4pt} % Size of the footer rule

% \setlength\parindent{0pt} % Removes all indentation from paragraphs

%----------------------------------------------------------------------------------------
%   TITLE PAGE
%----------------------------------------------------------------------------------------

\title{
\vspace{1in}
\textmd{\textbf{AC209a Data Science Project: Data Science with User Ratings and Reviews}}\\
% \normalsize\vspace{0.1in}\small{Due\ on\ \hmwkDueDate}\\
% \vspace{0.1in}\large{\textit{\hmwkClassInstructor\ \hmwkClassTime}}
\vspace{2cm}
}

\author{\textbf{Andrew, Sophie, Reiko, Nick}}
\date{\today} % Insert date here if you want it to appear below your name

%----------------------------------------------------------------------------------------

\begin{document}

\maketitle

\section*{Introduction}
	Note to everyone: I am just spit-balling some things here for possible use later on. Please don't take this as set in stone but it might be a useful start. Feel free to add comments or change things.

	Modern technologies unleash the possibilities for customising user experiences, products and thereby increase customer satisfaction by personalising content
	%https://www.analyticsvidhya.com/blog/2015/08/beginners-guide-learn-content-based-recommender-systems/
	based on the past preferences of a specific user. Content based recommendation systems can further be extended to not only providing customers with content that they are most likely to enjoy and associate with but also for advertisment purposes. One could argue that it is only worthwhile advertising the right type of product and content to the right type of person. If this is not done the advertisement company not only runs the risk of alienating potential customers, but at a best case scenario simply wastes the time and money placed into delivering the specific advertisment content to the specific customer. It has been shown (\textit{insert one of out references here}b) that potential customers are more likely to respond to content that directly appeals to them. Such is the power of a system that is able to make these structured connections from an unstructured source of data, such as user reviews and ratings.

	In this project, we aim to achieve the following:
	\begin{enumerate}
		\item Select and prepare a suitable database of user ratings and reviews from ... TODO
		\item Qualitatively, and quantitatively evaluate features and metrics upon which to train a statistical learning model.
		\item Implement and evaluate the following prediction and recommendation engines: simple user rating prediction, content based sentiment and recommendation, clustering of product and collaborative filtering.
		\item Decide upon the most appropriate recommendation based system for a particular setting and discuss some ethical considerations to be evaluated when making content based recommendations.
	\end{enumerate}

\section*{Project Description}

\section*{Literature Review - Paper 1}

\section*{Literature Review - Paper 2}

\end{document}












