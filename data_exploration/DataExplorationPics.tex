\documentclass[11pt]{article}
\usepackage{fullpage,amsmath,amsfonts,mathpazo,microtype,nicefrac,graphicx,verbatimbox,listings,hyperref,enumitem,amssymb,float,fancyhdr,caption,subcaption}
\DeclareGraphicsExtensions{.pdf,.eps,.png}

% Margins
% \topmargin=-0.45in
% \evensidemargin=0in
% \oddsidemargin=0in
% \textwidth=6.5in
% \textheight=9.0in
% \headsep=0.25in

% \linespread{1.1} % Line spacing

% % Set up the header and footer
% \pagestyle{fancy}
% \lhead{\hmwkAuthorName} % Top left header
% \chead{\hmwkClass\ (\hmwkClassInstructor\ \hmwkClassTime): \hmwkTitle} % Top center header
% \rhead{\firstxmark} % Top right header
% \lfoot{\lastxmark} % Bottom left footer
% \cfoot{} % Bottom center footer
% \rfoot{Page\ \thepage\ of\ \pageref{LastPage}} % Bottom right footer
% \renewcommand\headrulewidth{0.4pt} % Size of the header rule
% \renewcommand\footrulewidth{0.4pt} % Size of the footer rule

% \setlength\parindent{0pt} % Removes all indentation from paragraphs

%----------------------------------------------------------------------------------------
%   TITLE PAGE
%----------------------------------------------------------------------------------------

\title{
\vspace{1cm}
\textmd{\textbf{AC209a Data Science Project: Data Science with User Ratings and Reviews}}\\
% \normalsize\vspace{0.1in}\small{Due\ on\ \hmwkDueDate}\\
% \vspace{0.1in}\large{\textit{\hmwkClassInstructor\ \hmwkClassTime}}
}

\author{\textbf{Andrew Ross, Sophie Hilgard, Reiko Nishihara, Nick Hoernle}}
\date{\today} % Insert date here if you want it to appear below your name

%----------------------------------------------------------------------------------------

\begin{document}

\maketitle

\section*{Data Source}

We downloaded the data from the `Yelp Dataset Challenge' (\url{https://www.yelp.com/dataset_challenge}). The data contain in total 2.7M reviews from 687K users for 86K businesses. Business data consist of 15 features including ID, category of business (e.g., fast food, restaurant, nightlife, etc.), city, full address, operation hours, latitude, longitude, review count, and stars earned. User data consist of 11 features including average stars, compliments, elite, number of fans, IDs of his/her friends, name, review count, vote categories, and the month start a yelp review. So in terms of data that would be useful to recommendation systems, we have everything we need to construct a utility matrix (users, businesses, and ratings from 1-5 stars), in addition to a great number of additional interesting features that could be helpful either for recommendation or for general analysis to answer broader questions of interest.

\section*{Data Exploration}

\subsection*{Overview Exploration}

\par A simple inspection of the businesses data shows that useful information such as the business categories, latitude and longitude and the average ratings given to a business are all easily available. Similarily, for each user we have information about the number of reviews that they have given, the average rating that they give and the date that they have been `yelping' since.

\begin{figure}[H]
\centering
\includegraphics[width=0.7\textwidth]{./ac209/business-dataframe-simplified.png}
\caption{Head of Businesses Dataframe}
\end{figure}

\begin{figure}[H]
\centering
\includegraphics[width=0.7\textwidth]{./ac209/user-dataframe-simplified.png}
\caption{Head of Users Dataframe}
\end{figure}

A simple summary description of these dataframes is then presented below:

\begin{figure}[H]
\centering
\includegraphics[width=0.9\textwidth]{./ac209/bizdescribe.png}
\caption{Summary statistics of the Businesses Dataframe}
\end{figure}

\begin{figure}[H]
\centering
\includegraphics[width=0.9\textwidth]{./ac209/userdescribe.png}
\caption{Summary statistics of the User's Dataframe}
\end{figure}

\par We see from the summary data that most reviewers do not have fans and wrote a small number of reviews. Concretely, the median number of reviews given by a user is 5 yet the mean is 25. This suggests drastically right skewed data (which is intuitive as there is a lower bound of 0 on the number of reviews that a user can give). The inter-quartile range, IQR, for user rating was 3.2 - 4.6, again suggesting that most users rate businesses higher than the midpoint rating of 2.5. Similarly, we see that the businesses receive a mean rating of 3.69, with a mean count of 34.35. The IQR for review count for businesses is 5 to 26, again suggesting that a majority of businesses have a small number of reviews. \\

\par The review counts (number of reviews given by a user and number of reviews received by a business) are dramatically right skewed. We thus, omit the outliers in the below plot to understand the distribution of the $10^{th}$ to $90^{th}$ percentiles for review counts. We still see a hugely right skewed dataset, again intuitively it is more common for many users to rate a few number of businesses and it is more common for many businesses to receive a low number of reviews.

\begin{figure}[H]
\centering
\includegraphics[width=0.9\textwidth]{./ac209/user-and-business-review-hists-fixed.png}
\caption{Exponentially distributed count of reviews received per business (left) and count of reviews given per user (right)}
\end{figure}

% \par Figures for the count of reviews per user and count of reviews per business show a unimodal distribution with fatter tail in the right side. Most reviewers tend to write a small number of reviews, and therefore most businesses got a small number of reviews. A question that arises from this plot: for the businesses that receive a high number of ratings, are the ratings more or less positive than the businesses that receive a low number of ratings? The same question can be asked of the users.

\subsection*{Time based Exploration}
It was interesting to understand the distribution of the numbers of reviews over time for the yelp data.

\begin{figure}[H]
\centering
\includegraphics[width=0.9\textwidth]{./ac209/timeseries.png}
\caption{Number of reviews per day on yelp over time}
\end{figure}

We see that the number of reviews that are submitted on the Yelp platform over time is increasing dramatically from when Yelp opened in 2004 to present, were approximately 2500 reviews are submitted per day. Analyzing the number of reviews that are submitted over the course of a year shows that June and July are popular months while November and December show a relevantly lower number of reviews. An interesting point for further exploration is if the business type shows different `high' and `low' seasons and if the popularity of a business or a category of businesses can be tracked over time.

\subsection*{Location based Exploration}

We see that we are given 10 cities from the yelp data. From the metatdata about this database we know that these correspond to (the corresponding number of businesses was calculated for each city):
\begin{itemize}
	\item U.K.: Edinburgh (3480)
	\item Germany: Karlsruhe (1074)
	\item Canada:
		\begin{itemize}
			\item Montreal (5592)
			\item Waterloo (530)
		\end{itemize}
	\item U.S.:
		\begin{itemize}
			\item Pittsburgh (4088)
			\item Charlotte (7160)
			\item Urbana-Champaign (807)
			\item Phoenix (36505)
			\item Las Vegas (23598)
			\item Madison (3067)
		\end{itemize}
\end{itemize}

For the locations part of this study we will therefore narrow the focus to Las Vegas and Phoenix due to the large number of businesses in these cities and the expected restaurant culture that is a perception that is given of these cities.\\

\par For Phoenix the various categories were extracted from the data and the top business categories (by count) are shown below:
\begin{itemize}
	\item `Restaurants', 9428
	\item `Shopping', 5424
	\item `Food', 3637
	\item `Beauty \& Spas', 3603
	\item `Home Services', 3466
	\item `Health \& Medical', 3420
	\item `Automotive', 2629
	\item `Local Services', 2119
	\item `Nightlife', 1599
	\item `Active Life', 1551
\end{itemize}

We explored any trends in these categories and plotted the data by location. We also color coded the businesses by their average rating (with brown being the highest rating of 5 and gray being the lowest rating of 1). Restaurants are evenly distributed throughout Phoenix with an overwhelming number of highly rated businesses. The other business types are more evenly spread out but the Nightlife category shows a clear `hub' in the city center. Unfortunately, we see no clear areas that are high rated vs low rated areas. This is a specific point of interest (i.e. do certain areas come into and out of popularity) and thus will be investigated further throughout the course of this project.

\begin{figure}[H]
\centering
\includegraphics[width=1.1\textwidth]{./ac209/phxstarsbycategorylocation.png}
\caption{Chart showing the top 10 most popular business category plotted by location and colour coded by average review}
\end{figure}

\subsection*{Exploring the "Stars"}

Since a major focus of this project is on recommendations, we want to dive deeper into the star ratings data and see how we might use it to compute both baseline rankings for businesses and users (which was a major component of the model that won the Netflix challenge), as well as user-specific modifications. First, let's just look at the overall distribution of stars that users give to businesses, as well as the range of stars each user gives / each business receives:

\begin{figure}[H]
\centering
\includegraphics[width=0.9\textwidth]{./ac209/star-ranges-labeled.png}
\caption{Plot of the average and range of ratings that a user gives and the range of ratings that a business receives.}
\end{figure}

Here we see that users typically rate businesses highly (with a clear mode being 5 stars). However, we also note that businesses have a high range of votes (a mode of 4 indicates that users differ in their opinions). The figure also shows that users did not change their rating schema based on the business, and the majority of users rate all businesses within 1 point of each other. When we limit our selection to users who have given more than 15 reviews, this distribution changes a little bit:

\begin{figure}[H]
\centering
\includegraphics[width=0.9\textwidth]{./ac209/star-ranges-more-than-15-reviews.png}
\caption{}
\end{figure}

In particular, users are more likely to give 4s and less likely to give 5s and 1s. This is remniscent of the effect in the Netflix paper, when users who were giving more than one review per day were likely to give less extreme reviews (which was a major way they reduced their MSE that final amount to win). We can actually investigate the exact same phenomenon:

\begin{figure}[H]
\centering
\includegraphics[width=0.9\textwidth]{./ac209/more-than-one-per-day-dist.png}
\caption{Difference in stars when giving one or many reviews per day}
\end{figure}

When users give multiple reviews on a day, they tend to be more moderate in their ratings. If users are only giving one review per day, it might be because they just had a very positive or very negative experience on that day, and immediately log on to rate the business. Note that the fraction of users who only give one review per day has changed over the course of Yelp's history:

\begin{figure}[H]
\centering
\includegraphics[width=0.5\textwidth]{./ac209/prevalenceofratingsovertime.png}
\caption{Prevalence of ratings over time}
\end{figure}

A natural question to ask is whether users who give one review per day actually give more extreme reviews, or whether users in general have started giving more extreme reviews, and there are just more users who give one review per day. However, even when we conditioned on short time slices, we saw the same effect. A more extreme example is to condition on number of fans:

\begin{figure}[H]
\centering
\includegraphics[width=0.9\textwidth]{./ac209/lotsoffans.png}
\caption{Difference in stars between users with >50 and <=50 fans}
\end{figure}

If we condition on how many fans the user has, we find that that also plays a significant role in how extremely they rate users. However, we also discovered that even for users with more than 50 fans, when we condition on whether \textit{they} were giving one review per day or many, they were also more extreme when giving just one review (although the effect was less strong). This will be important to consider when computing a baseline model for business likeability.

\par Another issue in computing baselines is that we do have sparse data. To address this, we are considering adopting an approach similar to the movie recommendation homework, where we begin with a Beta prior assumption that every business has an average rating of 2.5 stars. More concretely, we introduced a 0 to 1 scale, assumed a Beta(8,8) prior, and added the ratings to rank all business, which gave us the following distribution of posterior "business ratings:"

\begin{figure}[H]
\centering
\includegraphics[width=0.5\textwidth]{./ac209/business-ratings-with-title.png}
\caption{Chart showing the top biased businesses}
\end{figure}

\par To give an intuition for what the highest and lowest ranked businesses are, we can just examine the top 5 of each:

\begin{table}[H]
\centering
\caption{Examples of low and highly ranked businesses (w/ rounded average stars, number of reviews, and whether they have since closed)}
\label{my-label}
\begin{tabular}{|l|l|l|l||r|r|r|r|}
	\hline
  \textbf{Lowest Ranked} & *s & \# & Closed & \textbf{Highest Ranked} & *s & \# & Closed \\ \hline
  A Victory Inn & 1 & 348 & 		 				      & Blue Chip Auto Glass & 5 & 121 & \\ \hline
  OnTrac & 1 & 28 & x                       & Lockaid USA & 5 & 151 &   \\ \hline
  Monitronics Security & 1 & 22 &            & Stell Roofing & 5 & 124 &  \\ \hline
  Anjile Cleaning Service & 1 & 26 & x  & Simply Skin Las Vegas & 5 & 133 &  \\ \hline
  Website Backup & 1 & 63 &                  & Khina Eyebrow Threading & 5 & 105 &  \\ \hline
\end{tabular}
\end{table}

\par Although the average stars here are rounded (a limitation in the summary statistics provided by the dataset), we can see that businesses with lots of high ratings rise to the top and businesses with lots of low ratings sink to the bottom. Also, some of those poorly-rated businesses are now closed, which might be interesting to explore further. How well can we use rating data to predict whether a business closes?

\par We are also interested in determining the effect that location plays in determining ranking. When we locate businesses on top of the actual map, we see that highly-rated businesses tend to locate on highways. This suggests that the accessibility is important to earn positive reviews as well as reviews itself, especially when the businesses are not located in the center of the city.
\begin{figure}[H]
\centering
\includegraphics[width=0.9\textwidth]{./ac209/mostreviewsbylocationlv.png}
\caption{Geographic location of top-rated businesses}
\end{figure}

\subsection*{Network Analysis}
\par We conducted a network analysis to see the topology and relationship among businesses based on user-rated 'stars'. When the star-ranking is positively correlated between two businesses, they are connected with a red edge in the business network. When the star-ranking is negatively correlated between two businesses, they are connected with a blue edge. We used Spearman's ranked correlation analysis to calculate correlation coefficients.
\begin{figure}[H]
\centering
\includegraphics[width=0.9\textwidth]{./ac209/networkanalysis.png}
\caption{Business correlation network based on user-rated stars in Phoenix.}
\end{figure}
\par This preliminary analysis only included 100 businesses (to clearly see the network structure) located in Phoenix. The node in the network figure indicates a business ID, and an edge represents a connection between businesses. The above figure shows that some businesses are highly connected with others, and there are businesses not connected (i.e., businesses were not rated by users). In the next step, we will further focus on a highly-rated businesses and specific type of businesses (e.g., restaurant).


\end{document}

