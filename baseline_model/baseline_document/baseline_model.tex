\documentclass[11pt]{article}
\usepackage{fullpage,amsmath,amsfonts,mathpazo,microtype,nicefrac,graphicx,verbatimbox,listings,hyperref,enumitem,amssymb,float,fancyhdr,caption,subcaption}
\DeclareGraphicsExtensions{.pdf,.eps,.png}

% Margins
% \topmargin=-0.45in
% \evensidemargin=0in
% \oddsidemargin=0in
% \textwidth=6.5in
% \textheight=9.0in
% \headsep=0.25in

% \linespread{1.1} % Line spacing

% % Set up the header and footer
% \pagestyle{fancy}
% \lhead{\hmwkAuthorName} % Top left header
% \chead{\hmwkClass\ (\hmwkClassInstructor\ \hmwkClassTime): \hmwkTitle} % Top center header
% \rhead{\firstxmark} % Top right header
% \lfoot{\lastxmark} % Bottom left footer
% \cfoot{} % Bottom center footer
% \rfoot{Page\ \thepage\ of\ \pageref{LastPage}} % Bottom right footer
% \renewcommand\headrulewidth{0.4pt} % Size of the header rule
% \renewcommand\footrulewidth{0.4pt} % Size of the footer rule

% \setlength\parindent{0pt} % Removes all indentation from paragraphs

%----------------------------------------------------------------------------------------
%   TITLE PAGE
%----------------------------------------------------------------------------------------

\title{
\vspace{1cm}
\textmd{\textbf{AC209a Data Science Project: Data Science with User Ratings and Reviews}}\\
% \normalsize\vspace{0.1in}\small{Due\ on\ \hmwkDueDate}\\
% \vspace{0.1in}\large{\textit{\hmwkClassInstructor\ \hmwkClassTime}}
}

\author{\textbf{Andrew Ross, Sophie Hilgard, Reiko Nishihara, Nick Hoernle}}
\date{\today} % Insert date here if you want it to appear below your name

%----------------------------------------------------------------------------------------

\begin{document}

\maketitle

\section*{Recommendation Baselines}

\par (Mention we limited to restaurants and bars in Phoenix and Las Vegas, that we think we can do more interesting analyses if we keep things local. Maybe compare the sparsity of the utility matrix when you make it local vs. global. Mention we have a standard train and test set we use across models, and that we'll split train into actual train and validation)

\subsection*{Performance Metric}

\par Following the Netflix challenge (link) and the majority of similar systems we found (link, link) specifically for the Yelp dataset, we will use RMSE, or the square root of the average squared difference between our prediction and the true number of stars a user awards a restaurant. This performance metric has an intuitive interpretation as a kind of star standard deviation.

\subsection*{Simple Averaging Baseline}

\par We should make sure that graphlab's popularity recommender works by taking the global average of all ratings, then adding the business's baseline offset and the user's baseline offset. We should report the RMSE.

\subsection*{Matrix Factorization Baseline}

\par We should play a bit more with the hyperparameters and compare Vowpal Wabbit and graphlab.

\subsection*{Collaborative Filtering Baseline}

\par We should actually do a minimal version of this and report a better RMSE than 4.

\subsection*{Future Work For Recommendations}

\par Make our baselines time-dependent, blend our models, other stuff?

\section*{Other Stuff We Did}

\par Location-based analysis, networks?

\end{document}

